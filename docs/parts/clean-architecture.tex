\part{Clean Architecture}
Die Ordnerstruktur des API Projekts ist orientiert an den Schichten
der \textit{Clean Architecture}
mit Abhän\-gigkeiten von außen nach innen. Die konkret umgesetzten Schichten
sind in \autoref{fig-clean-architecture} zu sehen.

\begin{figure}[h!]
  \center
  \begin{tikzpicture}
    [ line width=0.75pt,
      circ/.style={ circle, draw, inner sep=0pt, outer sep=0pt },
      arr/.style={ {Bar[width=4.5pt]}-{Triangle[fill=white, width=5mm, length=3mm]}, double distance=3pt }]

    \node[circ, minimum width=8cm, fill=Cyan!10] (outer) at (0,0) {};
    \draw (outer.west) -- (outer.east);

    \node[circ, minimum width=5cm, fill=blue!25] (mid) at (0,0) {};
    
    \node[circ, minimum width=2cm, fill=blue!65, text=white] (inner) at (0,0) {Domain};

    \node [yshift=-0.8cm] at (outer.north) {API};
    \node [yshift=-0.6cm] at (mid.south) {Infrastructure};
    \node [yshift=-0.9cm] at (mid.north) {Application};
    
    \draw [arr] ($(mid.south) + (0,1)$) -- ($(inner.south) + (0,0.5)$);
    \draw [arr] ($(mid.east) + (0.5,0.8)$) -- ($(mid.east) + (-0.8,0.5)$);
    \draw [arr] ($(mid.east) + (0.5,-0.8)$) -- ($(mid.east) + (-0.8,-0.5)$);
  \end{tikzpicture}
  \caption{Clean Architecture}
  \label{fig-clean-architecture}
\end{figure}

\noindent
In der Anwendungsebene werden hauptsächlich Interfaces definiert, welche
durch die Infrastrukturebene implementiert werden. Das API Projekt
sollte ausschließlich diese Interfaces verwenden und nie direkt
ein Objekt der Infrastruktur instanziieren.
Um diese Einschränkung durchzusetzten, hilft es,
die Klassen dieser Ebene mit dem Schlüsselwort \texttt{internal} zu markieren.
Die Domäne enthält anwendungsübergreifende Bausteine, wie
Datenstrukturen, Entitäten, Enums und die Implementierung der Wegfinde-Algorith\-men.
Das Domänen Projekt ist unabhängig von den anderen Schichten.
Es gibt außerdem ein fünftes Projekt mit dem Namen \texttt{Contracts}. Hier
werden alle Verträge beschrieben, die ein Anwender mit der Schnittstelle
haben kann. Verträge sind Datentransferobjekte (engl. \textit{data transfer object} (DTO))
und werden mit dem Suffix \texttt{Dto} gekennzeichnet. Datentransferobjekte
tauchen nur unmittelbar im Bereich der Schnittstelle auf und
sollten nie direkt in anderen Teilen der Anwendung verwendet werden.
Durch Mapping werden DTOs in Objekte der Domäne umgewandelt und umgekehrt:
\begin{equation*}
  \text{DTO} \Longrightarrow \text{\textit{domain object}} \Longrightarrow \text{DTO}
\end{equation*}
Diese Abbildung sollte immer geschehen, auch wenn sich die Objekte
sehr ähnlich sehen. Der Mapping Code befindet sich auf der Infrastrukturebene.