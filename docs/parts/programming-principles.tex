\part{Programming Principles} \label{part:programming-principles}
Es wurde versucht, während der Entwicklung
verschiedene \textit{programming principles} einzuhalten.
Im Folgenden werden die aus der Vorlesung behandelten Muster untersucht und
beschrieben, wie sie im Projekt umgesetzt wurden.

\section{SOLID}

\subsection{Single Responsibility Principle (S)}
Es wird hier als Beispiel die \lstinline{PathfindingService} Klasse
beschrieben, welche in \autoref{code:pathfinding-service} zu sehen ist.
\begin{lstlisting}[caption={\lstinline{PathfindingService} Klasse},
    label={code:pathfinding-service}]
internal class PathfindingService : IPathfindingService
{
    public PathfindingResult @\colorMethods BreadthFirstSearch@(Grid grid)
    {
        return new BreadthFirstSearch(grid).ShortestPath();
    }

    public PathfindingResult @\colorMethods Dijkstra@(Grid grid)
    {
        return new Dijkstra(grid).ShortestPath();
    }

    public PathfindingResult @\colorMethods AStar@(Grid grid)
    {
        return new AStar(grid).ShortestPath();
    }
}
\end{lstlisting}
Die Klasse muss nur genau dann geändert werden,
wenn ein neuer Algorithmus hinzugefügt wird.

\subsection{Open\,/\,Closed Principle (O)}
Jede Wegfinde-Klasse besitzt eine Methode \inlinecode[\colorMethods]{Short\-estPath}
und muss das folgende Interface implementieren.
\begin{lstlisting}[caption={Wegfinde-Algorithmus Interface},label={code:i-pathfinding}]
public interface IPathfindingAlgorithm
{
    PathfindingResult ShortestPath();
}
\end{lstlisting}
Eine weitere Klasse, die Wegfinde-Funktionalität benötigt, ist dadurch nicht abhängig
von einer bestimmten Implementierung. Die folgenden zwei Zeilen sind äquivalent.
\begin{lstlisting}[caption={Abstraktion der Wegfinde-Algorithmen},
label={code:pathfinding-abstracttion}]
IPathfindingAlgorithm bfs = new BreadthFirstSearch();
IPathfindingAlgorithm dijkstra = new Dijkstra();
\end{lstlisting}
Neue Verfahren können hinzugefügt werden (\textit{open}), ohne das die Klassen,
die Wegfinde-Funktionalität verwenden,
angepasst werden müssen (\textit{closed}).

\subsection{Liskov Substitution Principle (L)} \label{sec:lsp}
Die umgesetzten Wegfinde-Algorithmen benötigen eine
Funktion, welche von einem gegebenen Knoten aus die nächsten zu besuchenden
Knoten auswählt. Die Methode wird \inlinecode[\colorMethods]{GetNeighbors}
genannt und ist definiert in der abstrakten Klasse \inlinecode[\colorClasses]{Grid},
wie im folgenden Programmausschnitt zu sehen ist.
\begin{lstlisting}[caption={\textbf{Grid} Entität},label={code:grid-entity}]
public abstract class Grid
{
    public abstract List<GridNode> GetNeighbors(GridNode node);
}
\end{lstlisting}
Subklassen, die von \inlinecode[\colorClasses]{Grid} ableiten, müssen diese Methode überschreiben.
Für den Algorithmus spielt es keine Rolle, wie die konkrete Umsetzung aussieht.
Zum Beispiel könnte eine Implementierung die horizontal liegenden Nachbarn liefern,
eine andere die diagonal liegenden und eine weitere beide. Die folgenden zwei Zeilen sind äquivalent.
\begin{lstlisting}[caption={Abstraktion des Gitters}]
    Grid grid = new HorizontalGrid();
    Grid grid = new DiagonalGrid();
\end{lstlisting}

\subsection{Interface Segregation Principle (I)}
Es können hier die beiden Interfaces gennant werden, welche das in \autoref{eq:mapping}
gezeigte Mapping beschreiben.
\begin{lstlisting}[caption={Request Mapper}]
public interface IPathfindingRequestMapper
{
    Grid MapPathfindingRequestDto(PathfindingRequestDto pathfindingRequestDto);
}
\end{lstlisting}
\begin{lstlisting}[caption={Response Mapper}]
public interface IPathfindingResponseMapper
{
    PathfindingResponseDto MapPathfindingResult(PathfindingResult pathfindingResult);
}
\end{lstlisting}
Anstatt ein großes Allzweck-Interface (wie z.\,B. \lstinline{IPathfindingMapper})
umzusetzen, werden zwei kleinere Interfaces verwendet, getrennt nach Anfrage und Antwort.

\subsection{Dependency Inversion Principle (D)}
Gelöst durch die in \autoref{part:clean-architecture} beschriebene Architektur.
API (\textit{high level}) und Infrastrukturebene (\textit{low level})
sind beide abhängig von den Abstraktionen auf der Anwendungsebene.

\section{GRASP}
\subsection{Information Expert} \label{sec:info-expert}
Für eine neue Aufgabe ist derjenige zuständig, der schon das meiste Wissen für die Aufgabe
mitbringt. Beispiel: Manhattan Distanz zwischen zwei Punkten bestimmen.
Diese Funktion ist Teil der \ePosition{} Entität,
da hier schon die Hälfte des benötigten Wissens vorhanden ist.
\begin{lstlisting}[caption={Manhattan Distanz}]
public record Position(int Row, int Col)
{
    public int ManhattanDistance(Position p)
    {
        var v = (this - p).Absolute;
        return v.Row + v.Col;
    }
}
\end{lstlisting}

\subsection{Geringe Kopplung}
Die Funktionsweise von Klassen sollte durch Interfaces beschrieben werden.
Klassen haben keine direkten Abhängigkeiten, sondern werden durch \ac{di}
im Konstruktor zur Verfügung gestellt.
Im Fall von einer Änderung kann so ein Austausch der
Implementierung mit minimalem Aufwand (eine Zeile) erfolgen.
Eine geringe Kopplung wird nach diesem Prinzip erreicht.

\subsection{Hohe Kohäsion}
Es werden für dieses Prinzip die Klassen der Wegfinde-Algorithmen genannt
und hier am Beispiel von \lstinline{BreadthFirstSearch} erklärt.
\newpage
\begin{lstlisting}[caption={\lstinline{BreadthFirstSearch} Algorithmus}]
public class BreadthFirstSearch : IPathfindingAlgorithm
{
    private readonly Grid _grid;

    public BreadthFirstSearch(Grid grid)
    {
        _grid = grid;
    }

    public PathfindingResult ShortestPath()
    {
        // ...
    }
}
\end{lstlisting}
Die Instanzvariable von \lstinline{BreadthFirstSearch}
wird in allen (in diesem Fall einer) Methode verwendet.