\part{Entwurfsmuster}
Wegen der zur Zeit eher geringen Komplexität des Projektes hat es sich
nicht ergeben, dass eine Umsetzung von Entwurfsmustern sinnvoll erscheint.
Es ist unklar, ob die erzwungene Verwendung eines Musters
die Lesbarkeit des aktuellen Programmcodes verbessern würde.
Auf Seiten der Unittests kann aber dennoch ein Beispiel gegeben werden.
Es wird hier eine Form des Erbauermusters verwendet, um
komplexe Objekte für den Einsatz in Testmethoden zu erzeugen.
Konkret handelt es sich um die \inlinecode[\colorClasses]{GridFactory}
Klasse, welche anhand einer Zeichenkettenmatrix \inlinecode[\colorClasses]{Grid}
Objekte erzeugt (\ref{code:grid-entity}).
Die statische Klasse besitzt eine Methode \lstinline{Produce}
und nimmt entgegen ein \lstinline{string[]} sowie den Typ des
zu erstellenden Gitters. Die gekürzte Version der Klasse ist in
\autoref{code:grid-factory} zu sehen.
\begin{lstlisting}[caption={{\inlinecode[\colorClasses]{GridFactory}} Klasse},
  label={code:grid-factory}]
public static class GridFactory
{
    public static Grid Produce(string[] stringGrid, GridType gridType)
    {
        // ...
        return grid;
    }
}
\end{lstlisting}
