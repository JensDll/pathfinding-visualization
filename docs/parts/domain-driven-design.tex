\part{Domain Driven Design}
Die Domänensprache des Projekts umfasst die folgenden Begriffe:
\begin{itemize}
      \item \textbf{Grid} $\sim$ Das Gitter auf dem der kürzeste Weg gesucht und
            angezeigt wird.
      \item \textbf{GridNode} $\sim$ Knoten (engl. \textit{node}),
            welcher eine Position auf dem Gitter beschreibt. Ein
            Gitter besteht aus mehreren Knoten. Knoten haben neben
            primitiven Werten wie Gewicht, außerdem die
            folgenden Eigenschaften:
            \begin{itemize}[topsep=0pt]
                  \item \textbf{GridNodeType} $\sim$ Der Typ des Knotens mit Werten
                        wie Start, Ziel, Wand oder Standard.
                  \item \textbf{Position} $\sim$ Die Koordinate des Knotens in der
                        Form $(Zeile,Spalte)$.
            \end{itemize}
\end{itemize}
Die meisten Begriffe wie \textbf{Grid} und \textbf{GridNode} werden
im Programmcode als Entitäten bezeichnet.
\textbf{Position} hingegen ist ein \ac{vo}. Es macht mehr Sinn Punkte anhand ihrer
Werte zu unterscheiden und nicht anhand des gleichen Verweises.
\acp{vo} können in C\# mit dem in Version 9.0 neu eingeführten
Datensatztyp (engl. \textit{record type}) sehr bequem definiert werden.
Methoden um die Wertgleichheit sicherzustellen,
werden durch den Compiler automatisch erzeugt.
Eine gekürzte Definition des \textbf{Position} \ac{vo} ist in \autoref{code:position}
zu sehen.
\begin{lstlisting}[caption={Der Datensatztyp einer Koordinate},label={code:position}]
public record Position(int Row, int Col);
\end{lstlisting}
Die meisten Datentransferobjekte werden ebenfalls als \textit{record type} definiert.
Eigenschaften von \acp{dto} sind \inlinecode[\colorKeyword]{readonly},
dies ist vor allem deshalb sinnvoll, da Verträge unveränderlich sein sollten
und nur durch Mapping in eine ggf. veränderliche Datenstruktur gebracht werden.
Wegfinde-Algorithmen sind Gegenstand des \textit{abstraction code} auf
der Domänenebene. Sie werden als Teil eines \textit{service}
zusammengefasst mit dem Namen \inlinecode[\colorClasses]{PathfindingService} und
über diesen aufgerufen. Da für das Projekt nicht wirklich
eine persistente Datenspeicherung nötig ist, werden keine
Repositories verwendet.